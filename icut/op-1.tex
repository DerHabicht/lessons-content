\documentclass[talking]{usafpaper}

\title{ICUT---OP-1: Introduction to CAP Communications}
\author{Capt Hawk}
\authorOffice{UT080/DC}
\authorPhone{480-434-7796}
\typist{rhh}
\date{1 DEC 2022}


\begin{document}
\maketitle

\begin{itemize}
    \item Communications in the modern world depends largely on the internet.
    \begin{itemize}
        \item Even landline and cellular telephones eventually get routed
        through some form of Voice over IP (VoIP).

        \item The internet is a worldwide network of nodes and backbones.
        \begin{itemize}
            \item Much of this is wired.
            \item Some of it ``wireless'' utilizing radio towers and
            satellites.
        \end{itemize}

        \item Even ``wireless'' technologies rely on some kind of
        infrastructure such as:
        \begin{itemize}
            \item cell or transmission towers, 
            \item WiFi hotspots, or
            \item satellite control centers.
        \end{itemize}

        \item Large scale disasters can compromise this infrastructure in one
        or more of a few ways (usually all of them together):
        \begin{itemize}
            \item destruction of physical components can cause ``degradation of
            service,''
            \item multiple users utilize cellular and internet technologies at
            the same time in a disaster to get information and contact loved
            ones, and
            \item response agencies often need to communicate large amounts of
            information that multi-user communications backbones are not
            designed to handle.
        \end{itemize}
    \end{itemize}

    \item To understand how we get around these problems when prosecuting CAP
    missions, let's go back to basics.
    \begin{itemize}
        \item We've known how to communicate with radio waves for a while.
        \begin{itemize}
            \item First artificial radio waves generated by Heinrich Hertz in
            1888.
            \item First practical radio build by Guglielmo Marconi in 1895.
        \end{itemize}

        \item But, what is a radio wave?
        \begin{itemize}
            \item Energy can move through the universe in electric or magnetic
            form, we refer to these paths as ``fields.''
            \item Energy moving along electric and magnetic fields usually move
            as waves.
            \begin{itemize}
                \item Usually this is both an electric wave and a magnetic
                wave.
                \item The waves propagate perpendicular to each other.
                \item The orientation of the electric wave is what we call the
                wave's ``polarization.''
            \end{itemize}
        \end{itemize}
        \item Waves of different sizes have different properties.
        \begin{itemize}
            \item Theroetically, there is no 
        \end{itemize}
    \end{itemize}

    \item Radio Law
    \begin{itemize}
        \item Communications Act of 1934
        \begin{itemize}
            \item Created a hard line between ``federal'' communications and
            everything else
            \item Created the Federal Communications Commission (FCC) to
            regulate ``everything else'',
            \item National Telecommunications and Information Administration
            (NTIA) created in 1978 to---among other things---regulate
            ``federal'' communications
        \end{itemize}

        \item Radio Services
        \begin{itemize}
            \item Radio regulations tend to be grouped by use
            \item These usage categories are called ``services''
            \item 
        \end{itemize}

        \item For the purposes of radio regulation, CAP is a federal agency
        \begin{itemize}
            \item Our frequencies are assigned by the NTIA to the Air Force
            \item USAF further allocates their frequencies to us
            \item We must comply with both NTIA and USAF regulations pertaining
            to radio use
        \end{itemize}
    \end{itemize}

    \item Radio Usage in CAP
    \begin{itemize}
        \item
    \end{itemize}
\end{itemize}


\end{document}

